\documentclass[12pt]{article}
\usepackage{anyfontsize}
\usepackage[margin =2cm]{geometry}
\usepackage{polski}
\usepackage[utf8]{inputenc}
\usepackage{titlesec}
\titlelabel{\thetitle.\quad} 
\usepackage{tabto}
\usepackage{graphicx}
\usepackage{amsmath}
\usepackage{multicol}
\usepackage{mathtools}

\usepackage{csvsimple}
\usepackage{pgfplots}

\usepackage[european, americanvoltages, RPvoltages]{circuitikz}
\usepackage{tikz}

\title{\underline{Wzory, równania i zależności}\\\textbf{z teorii sygnałów}}
\author{Łukasz Przystupa}
\date{\today}
\pgfplotsset{compat=1.18}

\usepackage{titling}
\renewcommand\maketitlehooka{\null\mbox{}\vfill}
\renewcommand\maketitlehookd{\vfill\null}


\DeclareMathOperator{\FT}{\xleftrightarrow[\text{ICFT}]{\text{CFT}}}
\DeclareMathOperator{\CFT}{\xleftrightarrow[]{\text{CFT}}}
\DeclareMathOperator{\ICFT}{\xleftrightarrow[\text{ICFT}]{}}

\begin{document}

    \maketitle
    \begin{center}
        całość opara na wykładach Sypki, Gałki z 2020 \\oraz na książce ,,Cyfrowe przetwarzanie sygnałów od teorii do zastosowań" Zieliński.
    \end{center}
    \thispagestyle{empty}
    \newpage

    \section{Transformacja $\mathcal{Z}$ -- Laplace ale cyfrowo}
    \tab Transformacja $\mathcal{Z}$ jest cyfrowym odpowiednikiem transformacji Laplace'a.
    Definicja:
        \begin{equation*}
            X(z) = \sum_{n = 0}^{N-1} x(n)\cdot z^{-n}
        \end{equation*}
        gdzie:
        \begin{description}
            \item x -- sygnał dyskretny o N liczbie próbek
            \item n -- numer kolejne próbki (od 0 do N-1)
        \end{description}
    
    % \subsection{Kilka transformat}
    %     \begin{align*}
    %         &x(n) & X(&z)\\
    %         \delta(n) &= \left\{\begin{array}{l c}
    %             1, & n=0\\
    %             0, & n\neq 0
    %         \end{array}\right.
    %         & 
    %         X(z) &= 1\\

    %         -            
    %     \end{align*}

\section{Odwrotna transformata $\mathcal{Z}$}
        \begin{equation*}
            a
        \end{equation*}

\end{document}