\section{Szereg Fouriera}
    Postać numeryczna:
    \begin{gather*}
        x_F = a_0 + \sum\limits_{n=-\infty}^{+\infty}a_n\cdot cos\left(2\pi n\right) +b_n\cdot sin\left(2\pi n\right)
    \end{gather*}

    \indent Postać okresowa:
    \begin{align*}
        c_n = \frac{1}{T}\cdot \int\limits_{t_0}^{t_0+T}x(t) && gdzie:\ T - okres\ x(t)\\
% 
        x(t) = \sum\limits_{n=-\infty}^{+\infty}c_n\cdot e^{+j2\pi nf_T t} && gdzie:\ f_T = 1/T 
    \end{align*}
    Sygnał musi spełniać warunki Dirichleta!!!


\section{Definicje różnych transformat}
    \subsection{Transformata Fouriera (CFT i ICFT)}
        \begin{multicols}{2}
            \begin{gather*}
                x(t) \xleftrightarrow[\text{aaa}]{\text{bbb}} X(f)\\
                X(f) = \int\limits_{-\infty}^{+\infty} x(t) e^{-j2\pi f t}  \,dt 
            \end{gather*}

            \begin{gather*}
                x(a t)\xleftrightarrow[\text{ICFT}]{\text{CFT}} \frac{1}{|a|} X(\frac{f}{a})\\
                x(t - t_0)\xleftrightarrow[\text{ICFT}]{\text{CFT}} X(f)e^{-2j\pi ft_0}\\
                \overline{x(t)} \xleftrightarrow[ICFT]{CFT} \overline{X(-f)}
            \end{gather*}
        \end{multicols}
        Należy wspomnieć że iloczyn skalarny jest niezależny od wybranej dziedziny:
        \begin{gather*}
            x(t)\circ(y) \xleftrightarrow[\text{ICFT}]{\text{CFT}} X(f) \circ Y(t)\\
            \Downarrow\\
            \int\limits^{+\infty}_{-\infty}x(t)\overline{y(t)} dt \ \textbf{=} \int\limits^{+\infty}_{-\infty}X(f)\overline{Y(f)}df
        \end{gather*}


    \subsection{Transformacja sygnału próbkowanego}
        \begin{gather*}
            x_p(t) = x(t)\cdot g_{\Delta t}(t) = \sum_{n=-\infty}^{+\infty}x(n\Delta t)\cdot\delta(t-\Delta t)\\
            \Downarrow\\
            X_p(f) = X(f) * G_{\Delta t}(f) = X(f) * \left[\frac{1}{\Delta t} \sum_{n=-\infty}^{+\infty}\delta(f-n\cdot f_p)\right]
        \end{gather*}

    \subsection{Transformacja Dyskretna (DTFT)}
         \begin{multicols}{2}
            \begin{gather*}
                x_p = \sum\limits_{n=-\infty}^{+\infty}x(n\Delta t)\cdot \delta(t-n\Delta t)
            \end{gather*}

            \begin{gather*}
                X(f) = \sum\limits_{n = -\infty}^{+\infty}x(n\Delta t) \cdot e^{-j2 \pi \frac{f}{f_{p}}\cdot n}\\
                gdzie:\ f_p =\ f\ probkowania
            \end{gather*}
         \end{multicols}

    \subsection{Transformacja Hilberta}
        \begin{multicols}{2}
            \begin{gather*}
                x(t) \xleftrightarrow[\text{IHT}]{HT} x^H(t)\\
                x^H(t) = -\frac{1}{\pi}  \cdot \int\limits_{-\infty}^{+\infty}\frac{x(\tau)}{\tau-t}\ d\tau
            \end{gather*}

            \indent Czyli:
            \begin{gather*}
                x^H(t) = \frac{1}{\pi\cdot t}*x(t) = h_H(t)*x(t)\\
                \Downarrow\\
                h_H(t) \FT -j \cdot sgn(f)
            \end{gather*}
        \end{multicols}
    % \newpage

        
