\section{Wzory Eulera}
% 
\begin{multicols}{2}
    \begin{gather*}
        e^{jx} = \cos{(x)} + j\sin{(x)}\\
        \sin{(x)} = \frac{e^{jx}-e^{-jx}}{2j}\\
        \cos{(x)} = \frac{e^{jx}+e^{-jx}}{2}\\\\
        % 
        \cosh{(x)} = \frac{e^x+e^{-x}}{2}
    \end{gather*}

    \begin{gather*}
        \sin{(\alpha+\beta)} = \sin{(\alpha)}\cos{(\beta)} + \cos{(\alpha)}\sin{(\beta)}\\
        \sin{(\alpha-\beta)} = \sin{(\alpha)}\cos{(\beta)} - \cos{(\alpha)}\sin{(\beta)}\\
        % 
        \cos{(\alpha+\beta)} = \cos{(\alpha)}\cos{(\beta)} - \sin{(\alpha)}\sin{(\beta)}\\
        \cos{(\alpha-\beta)} = \cos{(\alpha)}\cos{(\beta)} + \sin{(\alpha)}\sin{(\beta)}
        \\\\
        \sinh{(x)}=\frac{e^x-e^{-x}}{2}
    \end{gather*}
\end{multicols}


\section{Sygnał jako wektor}
    Iloczyn skalarny:
    \begin{equation*}
        <x, y> = \int\limits_{D} x(t) \cdot \overline{y(t)} dt
    \end{equation*}
    Norma (długość wektora):
    \begin{equation*}
        ||x(t)||^2 = <x(t), x(t)>
    \end{equation*}
    Metryka (odległość sygnałów):
    \begin{equation*}
        \rho(x,y) = ||x-t|| = \sqrt{<x(t)-y(t), x(t)-y(t)>}
    \end{equation*}

    \noindent
    Korelacja (jak bardzo sygnały są podobne):
    \begin{equation*}
        R_{xy}(t) = \int\limits_{-\infty}^{+\infty}x(\tau)\cdot \overline{x(\tau-t)}\ d\tau
    \end{equation*}

    \noindent
    Energia sygnału:
    \begin{equation*}
        Energia(x(t)) = ||X(t)||^2_{L^2} = \int\limits_D |x(t)|^2dt
    \end{equation*}

    \noindent
    Wektory są ortogonalne (czyli prostopadłe względem siebie (czyli liniowo niezależne)) jeśli:
    \begin{equation*}
        <x(t), y(t)> = ||x(t)|| \cdot ||y(t)|| \cdot cos(\alpha) = 0 \Leftrightarrow x\perp y
    \end{equation*}


    \subsection{Twierdzenie Percevala - o zachowaniu energii}
        \begin{gather*}
            \int\limits^{+\infty}_{-\infty}|x(t)|^2 dt\ = \int\limits^{+\infty}_{-\infty}|X(f)|^2df
        \end{gather*}
    \subsection{Twierdzenie o zmianie skali}
        \begin{gather*}
            \delta(a\cdot t) \FT \frac{1}{|a|}\cdot\delta(f)
        \end{gather*}
    \subsection{Twierdzenie o zachowaniu odległości}
        Jeżeli:
        \begin{gather*}
            x(t)\circ y(t) \FT X(f)\circ Y(f)
        \end{gather*}
        to:
        \begin{gather*}
            ||x(t) - y(t)|| = ||X(f) - Y(f)||
        \end{gather*}
    \subsection{Twierdzenie o zmianie fazy (lub przesunięciu w czasie)}
        \begin{align*}
            x(t-t_0) \FT X(f) \cdot e^{-2pi\cdot f \phi}
        \end{align*}
        dla sygnału dyskretnego:
        \begin{align*}
            x(n-n_0) \FT X(k) \cdot e^{-2pi \frac{kn_0}{N}}
        \end{align*}
        gdzie:
        \begin{align*}
            &n - \text{numer próbki od 0}&&\\
            &N - \text{ilość próbek}\\
            &k - \text{numer harmonicznej}\\
        \end{align*}
        
        

    \subsection{Splot}
        Definicja:
        \begin{gather*}
            y(t) = \int \limits _{-\infty}^{+\infty}x_1(\tau) x_2(t-\tau) \,d\tau\\
            y(t) = x_1(t)*x_2(t) 
        \end{gather*}

        \noindent
        Właściwości splotu:
        \begin{gather*}
            x_1(t)*x_2(t)\xleftrightarrow[ICFT]{CFT} X_1(f)X_2(f)\\
            x(t)*\delta(t-t_0) = x(t-t0)
        \end{gather*}