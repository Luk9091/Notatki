\section*{Indeksy i ich znaczenie!}
    \tab W świecie inżynierów niezwykłą wagę przykłada sie do jasności i jednoznaczności oznaczeń. 
    Także i w tym skrypcie postaram się abyś Ty drogi czytelniku, nigdy nie miał wątpliwości, o co mi chodziło!
    Dlatego na wstępie muszę zapoznać Cię z kilkoma podstawowymi zasadami:

    \begin{enumerate}
        \item Oznaczenia napięć i prądów na elementach aktywnych:
            \begin{figure}[!h]
                \centering
                \begin{circuitikz}
                    \draw
                        (0, 0) to[I, i=$I_{V_0}$, v=$U_{V_0}$, l=$V_0$] ++ (3, 0)
                    ;
                \end{circuitikz}
            \end{figure}
        \item Oznaczenia prądów i napięć na elementach odbiorczych:
            \begin{figure}[!h]
                \centering
                \begin{circuitikz}
                    \draw
                        (0, 0) to[R, i=$I_{R_0}$, v=$U_{R_0}$, l=$R_0$] ++ (3, 0)
                    ;
                \end{circuitikz}
                % \caption{}
            \end{figure}
        \item Prądu i napięcia stałe:
            \begin{align*}
                &\text{Napięcie: }  &U_{\text{S}}\\
                &\text{Prąd: }      &I_{\text{S}}
            \end{align*}
        \item Prądy i napięcia zmienne:
            \begin{align*}
                &\text{Napięcie: }  &u_{\text{z}}\\
                &\text{Prąd: }      &i_{\text{z}}
            \end{align*}
        \item Prądy i napięcia o składowej stałej i zmiennej:
            \begin{align*}
                &\text{Napięcie: }  &u_{\text{S}}\\
                &\text{Prąd: }      &i_{\text{S}}
            \end{align*}
    \end{enumerate}

\newpage