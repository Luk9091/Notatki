\documentclass[12pt]{article}
\usepackage{anyfontsize}
\usepackage[a4paper, landscape, margin=2cm]{geometry}
\usepackage{polski}
\usepackage{tabto}
\usepackage{enumitem}
\usepackage{amsmath}
\usepackage{amssymb}
\usepackage{multirow}
\usepackage{multicol}
\usepackage{setspace}
\usepackage{pdfpages}

\usepackage{tabularx}
\newcolumntype{C}{>{\centering\arraybackslash}X}
\newcolumntype{L}{>{\raggedleft\arraybackslash}X}
\newcolumntype{R}{>{\raggedright\arraybackslash}X}
\newcommand{\centerY}[2]{\multirow{#1}{*}{#2}}

\usepackage{wrapfig}

\usepackage{hyperref}
\hypersetup{
    colorlinks = true,
    urlcolor=blue,
    linkcolor= black
}

\usepackage{chngcntr}
\counterwithin{figure}{section}
\counterwithin{table}{section}
\numberwithin{equation}{section}

\usepackage{graphicx}
\graphicspath{{./Img/}}

\usepackage{csvsimple}
\usepackage{pgfplots}
\usepackage{pgfplotstable}
\pgfplotsset{compat= newest}


\usepackage{titlesec}
\titlelabel{\thetitle.\quad}
% \AddToHook{cmd/section/before}{\clearpage}

\usepackage[european, american currents, americanvoltages, RPvoltages, cute inductor]{circuitikz}
\usepackage{tikz}
\usetikzlibrary{shapes.geometric}
\ctikzset{
    logic ports=ieee,
    logic ports/scale=0.7,
}


\usepackage{listings}
\lstset{
literate=%
    {ą}{{\k{a}}}1
    {Ą}{{\k{A}}}1
    {ć}{{\'c}}1
    {Ć}{{\'{C}}}1
    {ę}{{\k{e}}}1
    {Ę}{{\k{E}}}1
    {ł}{{\l{}}}1
    {Ł}{{\L{}}}1
    {ń}{{\'n}}1
    {Ń}{{\'N}}1
    {ó}{{\'o}}1
    {Ó}{{\'O}}1
    {ś}{{\'s}}1
    {Ś}{{\'S}}1
    {ż}{{\.z}}1
    {Ż}{{\.Z}}1
    {ź}{{\'z}}1
    {Ź}{{\'Z}}1
}


\title{
    \includegraphics[width = 0.3\textwidth]{agh_logo.jpg}\\
    \textbf{Akademia Górniczo-Hutnicza w Krakowie}\\
    Wydział Informatyki, Elektroniki i  Telekomunikacji\\\vspace{2cm}
    \textbf{Raport z projektu}\\
    Title
}
\author{
    \begin{tabularx}{\textwidth}{l l}
    Autor: &Łukasz Przystupa\\
    Kierunek studiów: & Elektronika i Telekomunikacja\\
    \end{tabularx}
}
\date{\vspace{2cm}\today}

\usepackage{titling}
\renewcommand\maketitlehooka{\null\mbox{}\vfill}
\renewcommand\maketitlehookd{\vfill\null}

\begin{document}
    \begin{table}[!ht]
        \vspace{4cm}
        \centering
        Ilość błędów w zależności od wzmocnienia diody ADP\\
        \begin{tabular}{|c|c|c|c|c|c|c|c|}\hline
            $U_{VDD}$ & $M$ & $m_{1}$ & $m_{2}$ & $\delta_1$ & $\delta_2$ & $Q$ & $BER$\\
                 $[V]$  &     & $[mV]$    & $[mV]$    & $[mV]$       & $[mV]$       &     &\\\hline
            34      & 1     & 2.3   & -23       & 1.4   & 1.4   & 1.6 & 5.1e-3  \\\hline
            50      & 3.5   & 6.8   & -6.7      & 1.5   & 1.3   & 4.8 & 7.2e-7  \\\hline
            54      & 5.8   & 13.4  & -13.5     & 2.2   & 1.4   & 7.5 & 4.0e-14 \\\hline
            56.8    & 17    & 38.3  & -39.3     & 6.0   & 2.5   & 9.1 & 3.4e-20 \\\hline
            58.5    & 37    & 86.5  & -82.2     & 23.6  & 36.6  & 2.8 & 2.6e-3  \\\hline
        \end{tabular}
    \end{table}


    \begin{figure}[!ht]
        \centering
        \begin{tikzpicture}
            \begin{axis}[
                width = 0.8\textwidth,
                grid = both,
                grid style= dashed,
                xlabel= Time {$[ms]$},
                ylabel= Intensity{$[nW]$},
                xmin = 0, xmax = 0.7,
                title = {Moc w funkcji czas},
                % ymin = 0, ymax = 1200
            ]
                \addplot [blue, mark=*] table [x=Time, y=Power, col sep=comma]{Measure/measure.csv};
                \addplot [red] coordinates {(0, 1076) (1, 1076)};
                \node [above] at (axis cs: 0.7/2, 1076) {Ref level};
            \end{axis}
        \end{tikzpicture}
    \end{figure}
    \pagestyle{empty}

    \begin{figure}[!ht]
        \centering
        \begin{tikzpicture}
            \begin{axis}[
                width = 0.8\textwidth,
                grid = both,
                grid style= dashed,
                xlabel= Distance {$[km]$},
                ylabel= Gain {$[dB]$},
                xmin = -2, xmax = 65,
                title = {Tłumienie w funkcji długości światłowodu}
                % ymin = -30, ymax = 12.5
            ]
                \addplot [blue, mark=*] table [x=Distance, y=Gain, col sep=comma]{Measure/measure.csv};
                \node [above=0.3cm] at (10, -2)  {$-0.420\frac{dB}{km}$};
                \node [above=0.3cm] at (30, -6) {$-0.414\frac{dB}{km}$};
                \node [above=0.3cm] at (50, -10) {$-0.406\frac{dB}{km}$};
            \end{axis}
        \end{tikzpicture}
    \end{figure}



    \begin{figure}[!ht]
        \centering
        \begin{tikzpicture}
            \begin{axis}[
                width = 0.8\textwidth,
                grid = both,
                grid style= dashed,
                xlabel= Moc wejściowa {$[dBm]$},
                ylabel= Moc {$[dBm]$},
                legend style={at = {(0.9, 0.1)}, anchor = south east},
                title = {Nieliniowości w światłowodzie},
            ]
                \addplot [blue, mark=*] table [x=Pin, y=Pout, col sep=comma]{Measure/nlin.csv};
                \addplot [red, mark=*] table [x=Pin, y=Pret, col sep=comma]{Measure/nlin.csv};
                \legend{Moc na wyjściu, Moc wracająca na wejście};
            \end{axis}
        \end{tikzpicture}
    \end{figure}
\end{document}
