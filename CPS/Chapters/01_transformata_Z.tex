\section{Transformacja $\mathcal{Z}$ -- Laplace ale cyfrowo}
    \tab Transformacja $\mathcal{Z}$ jest cyfrowym odpowiednikiem transformacji Laplace'a.
    Definicja:
        \begin{equation*}
            X(z) = \sum_{n = 0}^{N-1} x(n)\cdot z^{-n}
        \end{equation*}
        gdzie:
        \begin{description}
            \item x -- sygnał dyskretny o N liczbie próbek
            \item n -- numer kolejne próbki (od 0 do N-1)
        \end{description}
    
    % \subsection{Kilka transformat}
    %     \begin{align*}
    %         &x(n) & X(&z)\\
    %         \delta(n) &= \left\{\begin{array}{l c}
    %             1, & n=0\\
    %             0, & n\neq 0
    %         \end{array}\right.
    %         & 
    %         X(z) &= 1\\

    %         -            
    %     \end{align*}

\section{Odwrotna transformata $\mathcal{Z}$}
        \begin{equation*}
            a
        \end{equation*}