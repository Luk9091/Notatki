\section{Tranzystory bipolarne}
    \subsection{Struktura tranzystory}
        \begin{figure}[!h]
            \centering
            \begin{circuitikz}
                \draw
                    (0, 2) node[draw, minimum width = 2cm, minimum height = 1cm](n+){$n^{+}$}
                    (0, 1) node[draw, minimum width = 2cm, minimum height = 1cm](p){$p$}
                    (0, 0) node[draw, minimum width = 2cm, minimum height = 1cm](n){$n$}

                    (n+) ++ (1.25, 0) node[]{C}
                    (p)  ++ (1.25, 0) node[]{B}
                    (n)  ++ (1.25, 0) node[]{E}

                    (n+) -- ++ (0, 0.5) to[short, i< =$I_C$] ++ (0, 0.5) -- ++ (2, 0) to[I, invert, l=$U_{CE}$] ++ (0, -4) node[ground]{}
                    (p)  -- ++ (-1,0) to[short, i<_=$I_B$] ++ (-1, 0) to[I, invert, a=$U_{BE}$] ++ (0, -2) node[ground]{}
                    (n)  -- ++ (0, -0.75) to[short, i>_=$I_E$] ++ (0, -0.25) node[ground]{}
                ;
                \draw
                    (8, 1) node[npn](Q){}
                    (Q.C) to[short, i< =$I_C$] ++ (0, 1) -- ++ (2, 0) to[I, invert, l=$U_{CE}$] ++ (0, -3.25) node[ground]{}
                    (Q.B) to[short, i<_=$I_B$] ++ (-1, 0) to[I, invert, a=$U_{BE}$] ++ (0, -1.5) node[ground]{}
                    (Q.E) to[short, i>_=$I_E$] ++ (0, -0.75) node[ground]{}
                    
                    (Q.C) node[right]{C}
                    (Q.B) node[above]{B}
                    (Q.E) node[right]{E}

                ;
            \end{circuitikz}
            \caption{Struktura wewnętrzna tranzystora NPN, schemat tranzystory}
        \end{figure}

        \textbf{\underline{Stało-prądowa}} zależności między prądami:
        \begin{gather}
            I_C \approx I_E\\
            I_C = I_{ES} \cdot (e^{\frac{U_{BE}}{U_T}}-1) \approx I_{ES} \cdot e^{\frac{U_{BE}}{U_T}}
        \end{gather}
        Dla prostych obliczeń można wykorzystać stało-prądowy współczynnik wzmocnienia $\beta$:
        \begin{equation}
            I_C = \beta \cdot I_B
        \end{equation}

    \subsection{Schemat mało-sygnałowy}
        \tab Układ mało-sygnałowy rozpatrujemy przy zwartych źródłach napięcia \underline{stałego} oraz rozwartych źródłach prądu \underline{stałego}!
        \begin{figure}[!h]
            \centering
            \begin{circuitikz}
                \draw
                    (0, 0) node[above]{$B$}
                    (0, 0) to[short, o-, i=$i_b$] ++ (2, 0) to[R, a=$r_{BE}$, -*] ++ (0, -2) 
                    
                    (8, 0) node[above]{$C$}
                    (8, 0) to[short, i=$i_c$, o-] (6, 0)
                    (6, 0) to[R, l=$r_{CE}$, *-*] ++ (0, -2)
                    (6, 0) to[short] ++ (-2, 0) coordinate(I_c0) to [I, -*] ++ (0, -2) coordinate(I_C1)
                    (I_c0) to[rmeter, /tikz/circuitikz/bipoles/length=1.75cm] ++ (0, -2)
                    (I_c0) ++ (-0.75, -1) node[rotate = 90]{$g_m\cdot i_b$}
                    
                    (0, -2) to[short, o-] ++(6, 0)
                    (6, -2) to[short, -o, i=$i_e$] ++(2, 0)
                ;
            \end{circuitikz}
            \caption{Schemat mało-sygnałowy, tranzystora NPN}
        \end{figure}
        Parametry układu:
        \begin{align}
            g_m = \frac{I_C}{U_T} && r_{BE} = \frac{\beta}{g_m}
        \end{align}
        \begin{equation*}
            r_{CE} = \frac{U_A}{I_C}
        \end{equation*}
        Wzmocnienie napięciowe tranzystory:
        \begin{equation}
            k_u = \frac{dU_o}{dU_i} = -g_m\cdot r_{CE}
        \end{equation}
        Wzmocnienie prądowe:    
        \begin{equation}
            k_i = \beta
        \end{equation}
    
    \subsection{Układ ze wspólnym emiterem (OE)}
        \begin{figure}[!h]
            \centering
            \begin{circuitikz}
                \draw
                    (0, 0) node[npn](Q){}

                    (Q.B) -- ++ (-1, 0) to[sV, a=$u_g$] ++ (0, -2) node[ground]{}
                    (Q.E) -- ++ (0, -1.25) node[ground]{}

                    (Q.C) to[short, *-o] ++ (1, 0) node[above]{$U_{out}$}
                    (Q.C) to[R, l=$R_C$] ++ (0, 2) -- ++ (2.5, 0) to[I, invert, l=$U_{CC}$] ++ (0, -2) node[ground]{}
                ;
                \draw
                    (6, 0) node[above]{$B$}
                    (6, 0) to[short, o-, i=$i_b$] ++ (2, 0) to[R, a=$r_{BE}$, -*] ++ (0, -2) 
                    
                    (13, 0) node[above]{$C$}
                    (14, 0) to[short, i=$i_c$, o-] (13, 0)
                    (11, 0) to[R, *-*] ++ (0, -2)
                    (10.8,-1) node[above, rotate = 90]{$r_{CE}$}
                    (13, 0) -- ++(-2, 0)
                    (12, 0) to[R, l=$R_C$, *-*] ++ (0, -2) 
                    (11.5, 0) to[short] ++ (-2, 0) coordinate(I_c0) to [I, -*] ++ (0, -2) coordinate(I_C1)
                    (I_c0) to[rmeter, /tikz/circuitikz/bipoles/length=1.75cm] ++ (0, -2)
                    (I_c0) ++ (-0.75, -1) node[rotate = 90]{$g_m\cdot i_b$}
                    
                    (6, -2) to[short, o-] ++(8, 0)
                    (13, -2) to[short, -o, i=$i_e$] ++(1, 0)
                ;
            \end{circuitikz}
            \caption{Schemat wzmacniacza OE i model mało-sygnałowy}
        \end{figure}
        \begin{gather}
            k_u = -g_m\cdot (R_C || r_{CE}) \approx -g_m \cdot R_C = \frac{I_C\cdot R_C}{U_T}\\
            k_{u_{max}} = -g_m \cdot r_{CE}\\
            k_i = \beta % o to też warto dopytać
        \end{gather}
        Niezwykle ważnymi parametrami są rezystancję:\\
        wejściowa: -- dopytać bo nie jestem pewien :C
        \begin{equation}
            r_{in} = \frac{u_{in}}{i_{in}} = r_{BE}
        \end{equation}
        wyjściowa:
        \begin{equation}
            r_{out} = \frac{i_{out}}{u_{out}} = R_C || r_{CE}
        \end{equation}

    \newpage
    \subsection{Układ ze wspólną bazą (OB)}
        \tab Dużo rzadziej spotykaną konfiguracją tranzystora, jest połączenie ze wspólną bazą.
        \begin{figure}[!h]
            \centering
            \begin{circuitikz}
                \draw
                    (0, 0) node[npn, rotate = 90, yscale = -1](Q){}

                    (Q.B) to[short, i=$i_b$] (0, -2) node[ground]{}
                
                    (Q.E) to[short, i=$I_E$] ++ (-1, 0) coordinate(in)
                    (in) to[R, a=$r_G$] ++ (-2, 0) to[sV, a=$u_{in}$] ++(0, -2) node[ground]{}

                    (in) to[short, *-o] ++(0, -0.5) coordinate(in)
                    (in) ++ (0, -1.5) node[ground]{} to[short, o-] ++ (0, 0) 
                    (in) ++ (0, -0.75) node[left]{$U_{in}$}

                    (Q.C) to[short, *-o] ++ (1, 0) coordinate(out)
                    (Q.C) to[R, l=$R_C$] ++ (0, 2) node[vcc]{$V_{CC}$}
                
                    (out) ++ (0, -2) node[ground]{} to[short, o-] ++ (0, 0)
                    (out) ++ (0, -1) node[right]{$U_{out}$}
                ;
                \draw [latex-, thick] (in) ++ (0, -0.25) -- ++ (0, -1);
                \draw [latex-, thick] (out)++ (0, -0.25) -- ++ (0, -1.5);

                \draw
                    (8, 0) coordinate (B) to[R, a=$r_{BE}$, *-, i=$i_b$] ++ (0, -2) node[ground]{}
                    (B) to[short, i=$I_E$] ++ (-1, 0) coordinate(in) to[R, l=$r_G$] ++ (-2, 0) to[sV, a=$u_{in}$] ++ (0, -2) node[ground]{}
                    (B) to[short] ++ (1, 0) coordinate(out) --++(0, -1) to[R, l=$r_{CE}$] ++(2, 0) to[short, -*] ++ (0, 1)
                    (out) to[short, *-] ++(0, 1) coordinate(I_c0) to[I, invert] ++ (2, 0) 
                    (I_c0) to[rmeter, /tikz/circuitikz/bipoles/length=1.75cm, l=$g_mi_b$] ++ (2, 0) -- ++ (0, -1) coordinate(out)
                    
                    (out) to[short, *-o] ++ (1, 0) node[above]{$U_{out}$}
                    (out) ++(0.5, 0) to[R, l=$R_C$, *-] ++ (0, -2) node[ground]{}
                ;
            \end{circuitikz}
            \caption{Schemat ideowy oraz model mało-sygnałowy tranzystora ze wspólną bazą}
        \end{figure}
        
        Układ w tej konfiguracji może wydawać się bezużyteczny, ponieważ wzmocnienie napięciowe:
        \begin{equation}
            k_u = \frac{u_{oout}}{u_{in}} = \frac{(g_m r_O)R_c}{R_C+r_O} \approx g_m\cdot R_C
        \end{equation}
        W zasadzie dokładnie takie samo jak w układzie OE\dots
        \begin{equation}
            k_i \approx 1
        \end{equation}
        Rezystancja wejściowymi:
        \begin{equation}
            r_{in} = \frac{1}{g_m} \approx \frac{r_{BE}}{\beta}
        \end{equation}
        Rezystancja wyjściowa:
        \begin{equation}
            r_{out} = R_C||r_{CE} \left(1 + \beta \frac{r_G}{r_{BE} + r_G}\right) = R_C || r_{CE}
        \end{equation}

        Mała wartość impedancji wejściowej sprawia, że układ jest rzadko stosowany w układach mało-sygnałowych. -- generalnie układ gówniany

    \newpage
    \subsection{Układ ze wspólnym kolektorem (OC) - wtórnik emiterowy}
        \begin{figure}[!h]
            \centering
            \begin{circuitikz}
                \draw
                    (0, 0) node[npn](Q){}

                    (Q.C) to[short, i<=$i_C$] ++ (0, 1) node[vcc]{$V_{CC}$}
                    (Q.B) to[R, l=$r_G$, i<=$i_B$] ++(-2, 0) to[sV, a=$u_G$] ++ (0, -2) node[ground]{}
                    (Q.E) to[R, *-, l=$R_E$] ++ (0, -2) node[ground]{}
                    (Q.E) to[short, -o] ++ (1, 0) node[right]{$u_{out}$}
                ;

                \draw
                    (4, 2) to[R, l=$r_g$] ++ (2, 0) to[R, l=$r_{BE}$] ++ (0, -2) coordinate(E)
                    (4, 2) to[sV, l=$u_{in}$] ++ (0, -2) node[ground]{}
                    (E) to[short, -*, i=$i_b$] ++ (1, 0) coordinate(E)
                    (E) to[short, -*, i=$i_e$] ++ (0, -1) coordinate(out) 
                    (out) to[short, -o] ++ (1, 0) node[right]{$u_{out}$}
                    (out) to[R, l=$R_E$] ++ (0, -2) node[ground]{}

                    (E) to[short, i<=$i_c$] ++(1, 0) coordinate(I_c0) to [I, invert] ++ (0, 2) coordinate(I_C1)
                    (I_c0) to[rmeter, /tikz/circuitikz/bipoles/length=1.75cm] ++ (0, 2)
                    to[short] ++ (2, 0) to[R, l=$r_{CE}$] ++(0, -2) node[ground]{}
                ;
            \end{circuitikz}
        \end{figure}