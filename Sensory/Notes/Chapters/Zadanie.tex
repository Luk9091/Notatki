\task
Narysuj i omów podstawowe układy pracy detektorów światła (IR, widzialnego, UV) opartych na fotodiodach:


\begin{figure}[!ht]
    \begin{subfigure}{0.35\textwidth}
        \centering
        \begin{circuitikz}
            \draw
                (0, 0) to[pD, i^= $I_{out}$] ++ (0, -2) to[R] ++ (0, -2) node[ground]{}
                (0, 0)  node[vcc]{Fotodioda}

                (4, -1) to[pD, invert, v= $V_{out}$] ++ (0, -2) node[ground]{}
                (4, -1)  node[vcc]{Panel}
            ;
        \end{circuitikz}
        \vspace{1.5cm}
    \end{subfigure}
% 
    \begin{subfigure}{0.6\textwidth}
        \centering
        \begin{tikzpicture}
            \begin{axis}[
                % width = 0.4\textwidth,
                grid = both,
                title = Charakterystyka prądowo napięciowa diody,
                xlabel = Napięcie,
                ylabel = Prad,
            ]
                \addplot[blue] table[x = Vd, y = Id, col sep = comma]{Measure/Diode.csv};
                \node at (-6, 0) [above right, align = center]{Zakres pracy\\ jako fotodioda};
                \node at (2, 0) [above, align = center]{Panel\\słoneczny};
            \end{axis}
        \end{tikzpicture}
    \end{subfigure}
\end{figure}

\begin{enumerate}
    \item Dioda w kierunku zaporowym - zachowuje się jako fotodioda.
    \item Dioda w kierunku przewodzenia to panel
\end{enumerate}

\task
Wymień i omów podstawowe parametry detektorów światła.

\begin{itemize}
    \item czułość
    \item długość fali
    \item prąd ciemny
    \item szum własny
    \item zakres pracy
    \item czas narastania
    \item wzmocnienie
    \item zależności temperaturowe
\end{itemize}


\task
Wymień i omów podstawowe parametry czujników temperatury
\begin{itemize}
    \item zakres pomiarowy - zakres temperatur jakie mogą być poprawnie mierzone za pomocą danego czujnika
    \item dokładność - różnica między temperaturą rzeczywistą a zmierzoną
    \item rozdzielczość - najmniejsza zmiana zauważona przez czujnik
    \item czas odpowiedzi - czas jaki czujnik potrzebuje aby ustalić wartość mierzoną
\end{itemize}

\newpage
\task
Narysuj i omów podstawowe układy pracy mierników temperatury opartych o termopary.

Termopara to czujnik temperatury zbudowany z dwóch różnych metali połączonych ze sobą w punkcie pomiarowym (punkt gorący).
Po drugiej stronie czujnika (zimne złącze) powstaje napięcie termoelektryczne proporcjonalne do różnicy temperatur.
\begin{figure}[!ht]
    \centering
    \begin{circuitikz}
        \draw
            (0, 0) node[draw, rectangle, minimum height = 2cm](termopara){Termopara}
            (6, 0) node[op amp](opAmp){}
            (opAmp.-) to[short, a=strona zimna] ++ (-3.7, 0)
            (opAmp.+) to[short, l=strona ciepła] ++ (-3.7, 0)

            (opAmp.out) to[adc,>] ++(2, 0) node[right, draw, rectangle, minimum width = 1cm, minimum height = 1cm]{$\mu C$}
        ;
    \end{circuitikz}
\end{figure}

\task
Wymień i omów podstawowe parametry czujników przyśpieszenia

\task
Wymień i omów podstawowe parametry czujników przesunięcia kątowego (żyroskopów)

\begin{itemize}
    \item zakres pomiarowy
    \item czułość
    \item rozdzielczość
    \item dokładność
    \item powtarzalność
    \item offset
    \item pasmo przenoszenia
    \item oś pomiaru
\end{itemize}

\task

\begin{wrapfigure}[4]{r}{0.5 \textwidth}
    \centering
    \begin{tikzpicture}
    \draw
        (0, 0) to[R = $R_{load}$] ++(0, -2)
            coordinate(A)
        to[R = $R_{meas}$, *-*] ++(0, -2)
            coordinate(B)

        (A) --++ (-2, 0) to[R = $R_{bias}$] ++ (0, -2) node[ground]{}
        (0, 0) node[vcc]{}
        (0, -4) node[ground]{}

        (3, -3) node[op amp, noinv input up](opAmp){}
        (A) -| (opAmp.+)
        (B) -| (opAmp.-)
        (opAmp.out) to[short, -o] ++ (0, 0) node[above]{$V_{I_{meas}}$}
    ;
    \end{tikzpicture}
\end{wrapfigure}

\noindent 
Narysuj i omów podstawowe układy pracy mierników prądu\\
Dla zakresu:
\begin{itemize}
    \item nA – uA - mA
    \item mA - 3A
    \item 10mA - 100A (500A)
\end{itemize}

\newpage
\task
Omów narażenia środowiskowe wpływające na działanie podzespołów i układów
elektronicznych
\begin{enumerate}
    \item Temperatura
    \begin{itemize}
        \item dryft napięcia półprzewodników
        \item zmiany pojemności i/lub rezystancji
        \item skrócenie żywotności elementu
    \end{itemize}
    \item Wilgotność
    \begin{itemize}
        \item prąd upływu
        \item korozja styków i wyprowadzeń
        \item zwarcia w obwodzie
    \end{itemize}
    \item Pył i brud
    \begin{itemize}
        \item problemy z chłodzeniem
    \end{itemize}
    \item Środki chemiczne
    \begin{itemize}
        \item przebicia, zwarcia,
        \item niszczenie obudów
    \end{itemize}
    \item Wibracje i wstrząsy
    \begin{itemize}
        \item pęknięcia, uszkodzenia struktur scalonych
        \item zwiększenie rezystancji połączeń (np. lutów)
    \end{itemize}
    \item Promieniowanie elektromagnetyczne:
    \begin{itemize}
        \item zaburzenia w pracy układów
        \item zmiany w pamięciach trwałych
        \item przegrzewanie się urządzeń
    \end{itemize}
    \item Wyładowania ESD
    \begin{itemize}
        \item zniszczenie elementów
        \item obniżenie żywotności
    \end{itemize}
    \item Ciśnienie pracy
    \begin{itemize}
        \item błędne odczyty
        \item obniżenie żywotności
    \end{itemize}
\end{enumerate}
\newpage

\task
\begin{figure}[!ht]
    \centering
    \begin{circuitikz}
    \draw
        (0, -1) node[draw, rectangle, minimum width = 2cm, minimum height = 2cm, align = center](A){$\mu C$}

        (3, 0) node[draw, rectangle, minimum width = 1cm, minimum height = 1cm, align = center](B){Generator\\impulsów}
        (7, 0) node[draw, rectangle, minimum width = 1cm, minimum height = 1cm, align = center](C){Nadajnik\\ultradźwiękowy}

        (12, -2) node[draw, rectangle, minimum width = 1cm, minimum height = 1cm, align = center](D){Odbiornik\\ultradźwiękowy}
        (9.5, -2) node[plain mono amp, scale = 0.5, rotate = 180](E){}
        (7, -2) node[draw, rectangle, minimum width = 1cm, minimum height = 1cm, align = center](F){Układ detekcji\\sygnałów}
        (3, -2) node[draw, rectangle, minimum width = 1cm, minimum height = 1cm, align = center](G){Układ pomiaru\\czasu}

        (B) -- (C)
        (D) -- (E.in)
        (E.out) -- (F) -- (G)
    ;
    \draw[-Stealth] (A) -- (B);
    \draw[-Stealth] (G) -- (A);
    \end{circuitikz}
\end{figure}