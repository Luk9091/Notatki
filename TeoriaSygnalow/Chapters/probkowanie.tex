 \newpage
 \section{Próbkowanie}
    \tab Próbkujemy sygnał pseudo funkcją grzebieniową:
    \begin{gather*}
        x_p(t) = \sum_{n=-\infty}^{+\infty}x(t)\cdot \delta(t-n\Delta t)
    \end{gather*}
    W dziedzinie Fouriera odpowiada to:
    \begin{gather*}
        X_p(f) = \frac{1}{\Delta t} \cdot \sum_{n=-\infty}^{+\infty} X(f-n\cdot f_p)\\
        gdzie:\ f_p = \frac{1}{\Delta t}
    \end{gather*}

    \noindent Aby odtworzyć sygnał należy:
    \begin{multicols}{2}
        \begin{gather*}
            X(f) = X_p(f)\cdot H(f)
        \end{gather*}
        
        Gdzie:
        \begin{gather*}
            H(f) = \Delta t \cdot\Pi\left(\frac{f}{f_p}\right)
        \end{gather*}
    \end{multicols}
    \begin{center}
        $H(f)$ - funkcja transmitancji idealnego filtru dolnoprzepustowego
    \end{center}
    Czyli, funkcja odtworzona w dziedzinie czasu ma postać:
    \begin{gather*}
        x_r(t) = \sum_{n=-\infty}^{+\infty} x(n\Delta t) \cdot sinc(\pi f_p \cdot
        (t-\Delta t))
    \end{gather*}
    % 
    Zgodnie z twierdzeniem \textbf{Niquista-Shannona} poprawnie spróbkowany sygnał zawsze jesteśmy w stanie odtworzyć!
    Zgodnie z w/w twierdzeniem, częstotliwość sygnału próbkującego powinna spełniać zależność:
    \begin{equation*}
        f_{sampling} \ge 2\cdot f_{\text{sygnału}}
    \end{equation*}
    W przypadku nie spełnienia tej zależności powstanie tak zwany \textit{aliasing} -- czyli sygnał o częstotliwości mniejszej od $\frac{f_{sampling}}{2}$!\\
    Dodatkowo jeśli częstotliwość sygnału będzie: $f_{\text{sygnału}} == \frac{f_{sampling}}{2}$ to wystąpi próbkowanie krytyczne, które wynik zależy od różnicy faz sygnału głównego i próbkującego.
    