\section{Szereg Fouriera}
    Postać numeryczna:
    \begin{gather*}
        x_F = a_0 + \sum\limits_{n=-\infty}^{+\infty}a_n\cdot cos\left(2\pi n\right) +b_n\cdot sin\left(2\pi n\right)
    \end{gather*}

    \noindent Postać zespolona:
    \begin{align*}
        x(t) = \sum\limits_{n=-\infty}^{+\infty}c_n\cdot e^{+j2\pi nf_T t} && gdzie:\ f_T = 1/T\\
        c_n = \frac{1}{T}\cdot \int\limits_{t_0}^{t_0+T}x(t) e^{-j2\pi nf_T\cdot t}\ dt && gdzie:\ T - okres\ x(t)\\
        \text{Więc w CFT:} && c_n = \frac{\Delta t}{T} X_0(n\cdot \frac{1}{T})
    \end{align*}
    \begin{center}
        Sygnał musi spełniać warunki Dirichleta!!!
    \end{center}

    \noindent Postać okresowa:
    \begin{align*}
        x(t) = \sum_{n=-\infty}^{+\infty} x_0(t-n\cdot T)
    \end{align*} 

\newpage
\section{Definicje różnych transformat}
    \subsection{Transformata Fouriera (CFT i ICFT)}
        \begin{multicols}{2}
            \begin{gather*}
                x(t) \xleftrightarrow[\text{aaa}]{\text{bbb}} X(f)\\
                X(f) = \int\limits_{-\infty}^{+\infty} x(t) e^{-j2\pi f t}  \,dt 
            \end{gather*}

            \begin{gather*}
                x(a t)\xleftrightarrow[\text{ICFT}]{\text{CFT}} \frac{1}{|a|} X(\frac{f}{a})\\
                x(t - t_0)\xleftrightarrow[\text{ICFT}]{\text{CFT}} X(f)e^{-2j\pi ft_0}\\
                \overline{x(t)} \xleftrightarrow[ICFT]{CFT} \overline{X(-f)}
            \end{gather*}
        \end{multicols}
        Należy wspomnieć że iloczyn skalarny jest niezależny od wybranej dziedziny:
        \begin{gather*}
            x(t)\circ(y) \xleftrightarrow[\text{ICFT}]{\text{CFT}} X(f) \circ Y(t)\\
            \Downarrow\\
            \int\limits^{+\infty}_{-\infty}x(t)\overline{y(t)} dt \ \textbf{=} \int\limits^{+\infty}_{-\infty}X(f)\overline{Y(f)}df
        \end{gather*}

    \subsection{Transformacja sygnału próbkowanego}
        \begin{gather*}
            x_p(t) = x(t)\cdot g_{\Delta t}(t) = \sum_{n=-\infty}^{+\infty}x(n\Delta t)\cdot\delta(t-\Delta t)\\
            \Downarrow\\
            X_p(f) = X(f) * G_{\Delta t}(f) = X(f) * \left[\frac{1}{\Delta t} \sum_{n=-\infty}^{+\infty}\delta(f-n\cdot f_p)\right]
        \end{gather*}

    \subsection{Transformacja Dyskretna (DTFT)}
         \begin{multicols}{2}
            \begin{gather*}
                x_p = \sum\limits_{n=-\infty}^{+\infty}x(n\Delta t)\cdot \delta(t-n\Delta t)
            \end{gather*}

            \begin{gather*}
                X(f) = \sum\limits_{n = -\infty}^{+\infty}x(n\Delta t) \cdot e^{-j2 \pi \frac{f}{f_{p}}\cdot n}\\
                gdzie:\ f_p =\ f\ probkowania
            \end{gather*}
         \end{multicols}

    \newpage
    \subsection{Transformacja Hilberta}
        \begin{multicols}{2}
            \begin{gather*}
                x(t) \xleftrightarrow[\text{IHT}]{HT} x^H(t)\\
                x^H(t) = -\frac{1}{\pi}  \cdot \int\limits_{-\infty}^{+\infty}\frac{x(\tau)}{\tau-t}\ d\tau
            \end{gather*}

            \noindent Czyli:
            \begin{gather*}
                x^H(t) = \frac{1}{\pi\cdot t}*x(t) = h_H(t)*x(t)\\
                \Downarrow\\
                h_H(t) \FT -j \cdot sgn(f)
            \end{gather*}
        \end{multicols}

    \subsection{Transformacja okienkowana}
        \tab Odpowiednik transmitancji Fouriera, pomnożonej przez okno:
        \begin{equation*}
            \int \limits_{-\infty}^{\infty} x(\tau) \cdot  w(\tau-t) \cdot e^{-j2\pi f \tau}d\tau
        \end{equation*}
        $w(t)$ - funkcja okienkowa. Funkcja, która poza przedziałem osiąga 0!
    \subsection{Transformacja Gabora}
        \tab To specjalny rodzaj transmitancji okienkowej, w której okno jest funkcją Gausa.
        \begin{multicols}{2}
            \begin{gather*}
                x(t) = \int \limits_{-\infty}^{+\infty} X^G(f, t)\cdot e^{+j2\pi \cdot f \cdot t}\ df
            \end{gather*}\hfill
            \noindent
            \begin{gather*}
                X^G(f, t) = \int \limits_{-\infty}^{+\infty}x(\tau)\cdot w(\tau-t)\cdot e^{-j2\pi\cdot f\cdot t}\ d\tau\\
                gdzie:\ w(t) = e^{-\pi\cdot f^2}
            \end{gather*}
        \end{multicols}

    \subsection{Transformacja falkowa}
    \begin{gather*}
        X_\psi(a, t) = \frac{1}{\sqrt{a}}\int\limits_{-\infty}^{+\infty}x(\tau)\cdot h_\psi\left(\frac{t-\tau}{a}\right)\ d\tau\\
        \Updownarrow\\
        x(t) = \frac{1}{c_\psi}\cdot \int\limits_{0}^{+\infty}\frac{1}{a^2\cdot\sqrt{a}}\cdot \int\limits_{-\infty}^{+\infty}X_\psi(a, \tau)h_\psi\left(\frac{\tau-t}{a}\right)\ d\tau\ da\\
        przy\ czym:\ c_\psi 2\cdot \int\limits_{0}^{+\infty}\frac{|H_\psi(f)|^2}{f}df < +\infty
    \end{gather*}
    $h_\psi(t)$ - falka sygnału. Funkcja, która poza przedziałem dąży do 0
    