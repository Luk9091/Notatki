\section{Wzmacniacze operacyjne}
    \begin{figure}[!h]
        \centering
        \begin{circuitikz}
            \draw
                (0, 0) node[op amp]{$k_0$}
            ;
        \end{circuitikz}
        \label{Wzmacniacz operacyjny}
    \end{figure}
    \subsection{Parametry}
    \begin{itemize}[label = -]
        \item $k_0$ -- wzmocnienie wzmacniacza w otwartej pętli sprzężenia zwrotnego
        \item PSRR -- współczynnik odrzucenia wpływu zasilania
            \begin{equation}
                PSRR = \frac{\Delta U_{out}}{\Delta U_{supply}}
            \end{equation}
        \item Slew rate -- szybkość narastania zboczy
            \begin{equation}
                SR = \left| \right|
            \end{equation}
        \item Wide bandwidth -- pole wzmocnienie
            \begin{equation}
                bandwidth = Gain\cdot Band
            \end{equation}
            Jeśli w nocie katalogowej znajduję się informacja o nominalnym polu wzmocnienia to należy rozumieć je jako iloczyn:
            \begin{equation}
                bandwidth = 1\frac{V}{V}\cdot Band
            \end{equation}
    \end{itemize}
    
